% THIS TEMPLATE IS A WORK IN PROGRESS

\documentclass{article}

\usepackage{hyperref}
\usepackage{fancyhdr}
\usepackage{ctex}

%\lhead{\includegraphics[width=0.2\textwidth]{nyush-logo.pdf}}
\fancypagestyle{firstpage}{%
  \lhead{《通信与网络》}
  \rhead{2022~2023年秋季学期\ \ 实验报告}
}

%%%% PROJECT TITLE
\title{\textbf{实验一:电平信道编译码}\\
        \Large \emph{实验报告}}

%%%% NAMES OF ALL THE STUDENTS INVOLVED (first-name last-name)
% \author{\href{mailto:author1@nyu.edu}{Author \#1 Name} (auth\#1 NetID) - \emph{auth\#1 major}\\ \href{mailto:author2@nyu.edu}{Author \#2 Name} (auth\#2 NetID) - \emph{auth\#2 major}}
\author{李沐阳、王铮、赵英竹、李智毅}


% \date{\vspace{-5ex} 2022年10月16日} %NO DATE
\date{2022年10月16日}


\begin{document}
\maketitle
\thispagestyle{firstpage}

\section*{Supervision}

xxx

\section*{卷积码的设计及分析}

\subsection*{卷积码译码}
分为软判决和硬判决译码。软判决译码需要进行viterbi译码,在viterbi算法中,将收到的电平与‘0’和‘1’代表的电平做内积,作为判定的度量值。硬判决译码先将收到的电平进行判决,之后进行viterbi译码。\\
viterbi算法:首先生成

\bibliographystyle{IEEEtran}
\bibliography{references}



\end{document}